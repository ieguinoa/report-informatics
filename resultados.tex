\chapter{Resultados}
\section{Propiedades de los sistemas de evaluación}
% ACA DEBERIA DECIR LAS CARACTERISTICAS DE LOS SISTEMAS Y CUALQUIER COSA DE LA EJECUCION QUE TENGA ALGO QUE VER CON EL SISTEMA QUE SE ESTÁ UTILIZANDO

Los tiempos de ejecuci\'on se midieron utilizando el reloj est\'andar de alta precisi\'on para Linux (\texttt{clock\_gettime}~\cite{LinuxDocumentation}), 


Claramente, el factor mas importante del sistema que afectará a la ejecución es el número total de partículas de éste. Los tamaños de sistemas utilizados fueron tres, y para 

Además del tamaño, hay otros factores del sistema que se tuvieron en cuenta para realizar una correcta evaluación y que deben ser 
Para tener en cuenta cualquier posible diferencia relacionada a los tipos de partículas, la asignación de tipos se hizo de forma aleatoria entre un total de 37 tipos existentes.

Además de las posiciones y tipos de particulas, es posible especificar velocidades iniciales....

En cuanto a los parámetros de ejecución se debe tener en cuenta que a medida que la simulación avanza, las partículas 
se deben tener en cuenta, entonces, dos aspectos que asocian la ejecucion con los parametros del sistema:
-las velocidades iniciales(ya descrito antes)
-el largo de la simulación y el uso de periocidad: es importante tener en cuenta que, si no se usa periodicidad, luego de cierto punto las particulas . 

Este efecto se verá en cualquier simulación independientemente de las propiedades del sistema inicial y, claramente es algo a tener en cuenta ya que afecta la correcta evaluación, la aproximación que usamos es aplicar condiciones periódicas de contorno en todas las simulaciones, independientemente del tamaño del sistema.
Al usar condiciones periodicas 
El tamaño de la caja es, entonces, otro factor importante . De esta forma, a menos que se especifique lo contraro, se define como tamaño de la caja al valor de cutoff/2.

Todos estos factores deben ser tenidos en cuenta a la hora de evaluar las implementaciones. 
, las situaciones mencionadas que afectan a la ejecucion sirven para demostrar propiedades de las implementaciones y como ejecuciones de control, para 
En las próximas secciones, cuando sea relevante, se especificarán las propiedades del sistema y de la ejecucion que se utilizaron.


\section{Performance}
Los sistemas detallados en la sección anterior se utilizaron para evaluar las distintas implementaciones en cuanto a performance..

\subsection{Evaluación de las implementaciones}

% PRIMER GRAFICO DEBERÍA IR EL TIEMPO DE EJECUCION DE LAS 3 IMPLEMENTACIONES PARA LOS DISTINTOS SISTEMAS (PERIODICO)
% GRAFICO DE BARRA CON LAS 3 IMPLEMENTACIONES AGRUPADAS EN 3(O 2) SISTEMAS DE PRUEBAS **PERIODICO -  CANTIDAD FIJA DE PASOS*
% *****PUEDO METER LO DE CPU TAMBIEN ACA  ****

\subsubsection{Efectos de la periodicidad}
% DESPUES EXPLICAR COMO AFECTA LA (NO)PERIODICIAD A LA EJECUCION Y MOSTRA EL GRAFICO DE TIEMPO DE EJECUCION DE LAS 3 IMPLEMENTACIONES PARA LOS DISTINTOS SISTEMAS (***NO** PERIODICO)

\subsubsection{Efectos del cutoff}
Se realiza una evaluación del efecto del cutoff en el tiempo de ejecución
% GRAFICO SCATTER DE LAS 3 IMPLEMENTACIONES vs AUMENTO EN EL CUTOFF


\subsection{Efectos del tamaño de bloque}


\section{Calidad numérica}

\subsection{Tamaños y límites de las tablas}
% GRAFICO DE ERROR ABSOLUTO vs NUMERO TOTAL DE ELEMENTOS DE LA TABLA

% GRAFICO
\subsection{Estabilidad de la simulación}
% Acá debería mostrar como la energia se mantiene constante cuando hago una simulacion larga (distintos valores de tamano de tabla	)


