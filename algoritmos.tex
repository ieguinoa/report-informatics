
\chapter{Algoritmos}


En este capítulo, los conceptos del m\'etodo de dinámica molecular presentados en la introducción son explicados en términos del algoritmo que los representa,
haciendo énfasis en el paralelismo intrínseco que poseen, intentando abstraerse de cualquier implementación.

En la primera sección se describen los conceptos del algoritmo base, describiendo todos los pasos de éste, en particular los detalles que no fueron abarcados en la introducción. 

En las siguientes secciones del capítulo se explican las modificaciones que este trabajo introduce y las propiedades que se deben tener en cuenta para implementarlas utilizando arquitecturas paralelas.

% Es importante destacar que las adaptaciones que se plantean y analizan en este trabajo constituyen una modificación puntual para realizar el calculo del potencial de Lennard-Jones y, por lo tanto, 
% tienen la capacidad de ser incluidas directamente en cualquier implementación existente y combinadas junto con cualquier otro método de optimización.
% Por otro lado, como se vió en el capitulo 1, existen pocas variantes para optimizar el cálculo de este potencial de no-unión y, dado que comprende el mayor aporte al costo computacional del método, 
% estas modificaciones pueden tener un gran impacto en la .


%Dado que el aporte a las interacciones que nos interesa es el correspondiente al potencial de L-J , este es el unico que se va a considerar. 
%Es decir, el campo de fuerzas estará definido exclusivamente por el potencial de Lennard-Jones que modela interacciones de no-unión. 
%De esta forma, el sistema evoluciona en el tiempo a partir de las velocidades iniciales de las particulas y las fuerzas que se derivan de este potencial, unicamente. 
% Sin embargo, usando el modelo de interacción resultante no será posible simular (de forma correcta) sistemas moleculares que incluyan enlaces químicos, ya que estos no estarán correctamente modelados por el campo de fuerzas. 
% El objetivo de esto es, en primer lugar, facilitar la implementacion de modificaciones/optimizaciones en el algoritmo relacionadas con el calculo del potencial de L-J.
% En segundo lugar, utilizando una implementacion propia se puede evaluar las ventajas y desventajas del uso de una tabla numerica de forma independiente a cualquier otra implementacion existente. 
% Se podrá verificar facilmente que las variaciones en el tiempo y en la precision de los resultados son producto exclusivo de las modificaciones implementadas en cada paso.




\section{Esquema general del algoritmo}

El algoritmo base sigue la especificación que se detallo en el capítulo 1(figura \ref{esquemaMD}). 

El primer paso de la iteración implica calcular la fuerza que actúa sobre cada partícula. 
El cálculo de la fuerza entre cada par de partículas responde a la siguiente ecuación: \begin{equation} \label{fuerza}\vec{F}(r)=\dfrac{\partial U(r)}{ \partial(r)}\end{equation}
                                                                                                                                                   
Por lo tanto, para realizar este cálculo en forma discreta, primero debemos conocer el valor de \textit{r}, es decir, la distancia entre cada par de partículas.
Este paso es claramente paralelizable ya que las posiciones son totalmente independientes entre si. El cálculo necesario es simplemente la resta entre los componentes de las coordenadas de cada partícula.
La única excepcion es cuando se están utilizando condiciones periódicas de borde. En este caso la distancia enttre cada par de partículas se debe calcular solo entre las imágenes mas cercanas.

A partir de las distancias entre las partículas, la ecuación \ref{fuerza} permite obtener la fuerza resultante de la interacción entre ellas, conociendo la función potencial.
Este valor corresponde a la fuerza en la dirección de $\vec{r}$, la dirección del vector distancia. 
Para poder obtener la suma total de las fuerzas que actúan sobre una partícula como resultado de interacciones con todos los demás elementos del sistema,
es necesario sumar todos los aportes. Esto implica sumar fuerzas con distintas direcciones, lo que se traduce en una suma de los componentes x,y,z de los distintos vectores. 
Por lo tanto, el cálculo de la fuerza total que actúa sobre una partícula se divide en tres pasos:

\begin{itemize}
\item En un primer paso se calcula el valor de la fuerza entre cada par de partículas de acuerdo a la ecuación \ref{fuerza}.

\begin{equation} 
  \dfrac{\partial(U(r)) }{ \partial(r)}  =  24\epsilon \bigg(\dfrac{{\sigma}^{6}}{{r}^{7}} - \dfrac{2{\sigma}^{12}} {{r}^{13}}\bigg)
\end{equation}


\item En un segundo paso se descompone esta fuerza en sus 3 componentes (x,y,z): \begin{equation}                                                                                
F_i=\vec{F}(r)\dfrac{\partial r}{\partial i}  \hspace{30pt} \text{para i= x,y,z}
                                                                                 \end{equation}

\item Por último se realiza la suma de los componentes para todas las fuerzas que actúan sobre una partícula: 
\begin{equation}
F_i=\sum F_i
\end{equation}

\end{itemize}



% CALCULO DE LA ACELERACION/CAMBIO EN LA VELOCIDAD
Una vez obtenida la fuerza que actúa sobre cada partícula(componetes x,y,z), se puede calcular la aceleración(a=cambio en la velocidad) resultante de ésta y actualizar la velocidad de la partícula de acuerdo a este cambio.
La velocidad estará dada por:
 \begin{equation} 
    v(t+dt)= v(t) + a\times dt =  v(t) + \bigg(\frac{F}{masa}\bigg)\times dt
% velocity[i] = Vt + ( (force[i]*dtx) / m[type] );
 \end{equation}

Nuevamente, la velocidad depende de propiedades de la partícula (velocidad anterior, fuerza y masa) y el cálculo es totalmente independiente para las distintas partículas.  


% PASO DE ACTUALIZACION DE CORDENADA
El proceso de actualización de coordenadas deberá calcular para cada partícula su nueva posición, la cual estará dada en función de las coordenadas anteriores, la velocidad que se calculó previamente en la iteración y el \textit{step} en el tiempo. 
La ecuación que se debe resolver en este paso es:
\begin{equation} 
    Pos_x(t+dt)= Pos_x(t) + (v_x \times dt)
% positions[i] = positions[i] + (velocity[i] * dtx);
\end{equation}

Claramente, este cálculo es totalmente independiente entre las partículas y por lo tanto puede hacerse completamente en paralelo.



Las limitaciones en este conjunto de pasos se deben a que tienen un orden de ejecución $N^2$. Aún en arquitecturas altamente paralelas como las GPUs, donde el número de procesadores crece continuamente, 
no es tan simple obtener el máximo de performance teórico de la arquitectura, principalmente debido a limitaciones en los accesos a memoria.
Es necesario, entonces, realizar aproximaciones extra para acercarse a este límite.




\section{Calculo usando tablas de valores del potencial }


Basandose en la forma funcional del potencial mostrada en el capítulo 1(figura \ref{lennardimage}), es razonable pensar que, utilizando una tabla conteniendo los resultados para un rango definido de valores de \textit{r}, y, 
aún sin tener ésta un tamaño excesivo, es posible utilizarla para obtener una estimación bastante aproximada de la interacción entre dos partículas. 
Utilizando esta tabla, para obtener el valor de la energia potencial de interaccion entre un par de particulas simplemente se recupera el valor asociado a la distancia(\textit{r}) que existe entre éstas(o el valor tabulado mas cercano).
Esta propiedad se deriva directamente de las características de la función y es independiente del esquema de dinámica molecular.


% Teniendo este tipo de valores almacenados es posible utilizarlos para calcular tanto las fuerzas resultantes (necesarias para la evolucion de la simulacion) como el valor de la enegia potencial total, el cual puede ser requerido por el usuario como resultado de la simulacion.

En la sección anterior se mostró como el algoritmo utiliza este potencial para obtener la interacción entre cada par de particulas y derivar la fuerza resultante. 
Para obtener el valor de esta fuerza es necesario conocer la derivada de la función potencial para la distancia correspondiente.
Utilizando la tabla descita previamente se puede obtener el valor de la derivada mediante la técnica de derivación numerica. 
Para esto se obtienen primero dos valores de potencial a partir de la tabla, correspondientes a distancias de $r\pm h$. 
% Donde x es un valor discreto definido en base a.....
La aproximación de la derivada se calcula entonces como:

\begin{equation}
 V'(r) \approx \frac{V(r+h)-V(r-h)}{2h}
\end{equation}

Utilizando estas aproximaciones sería posible simplificar considerablemente el cálculo de las fuerzas resultantes del potencial si se lo aplica dentro de un esquema de dinámica molecular.

La aproximación de la derivada por el método de derivación numérica entrega resultados aceptables con un determinado error. 
La utilización del método junto con el esquema de valores tabulados implica, entonces, una doble fuente de error. 
Una parte importante de aplicar esta aproximación será, entonces, asegurar la correctitud numérica de los resultados, 
principalmente de los efectos que se acumulan a lo largo de una simulación que involucra una gran cantidad de pasos y que genera la propagación de los errores.
El algoritmo de dinámica molecular utiliza la fuerza para modificar la posición de las partículas y, por lo tanto, el valor del potencial en la próxima iteración. 
De esta forma, los errores resultantes del método serán propagados a lo largo de una simulación que, normalmente, involucra miles de pasos.

Como se vió previamente, la ecuación del potencial de Lennard-Jones (ecuación \ref{lennardEquation}) depende directamente de \textit{r} y de los valores de $\epsilon$ y $\sigma$ de ambas partículas.
La tabla que se utilizará tendrá, entonces, tres dimensiones. En una dimensión el valor del potencial variará en función de \textit{r}.
Las otras dos dimensiones representan una matriz que contiene todas las combinaciones de pares $\epsilon$-$\sigma$.

Una ventaja adicional de tener tabulados los valores del potencial es que permite calcular muy fácilmente el valor de la energía potencial total asociada a una conformación puntual del sistema.
Si bien la salida principal del método, suele adjuntarse además, el valor de $E_tot$ asociado a la conformación cada cierta cantidad de pasos. Este valor es utilizado durante el proceso de análisis posterior.

Si bien el objetivo de este trabajo es optimizar la ejecución de las implementaciones sobre GPU, la solución aqui planteada puede ser implementada en distintas arquitecturas.
El esquema general del algoritmo se mantiene en todas las variantes. Las modificaciones estan centradas en la etapa de calculo del potencial, el contexto de este se mantiene siempre igual.
Se debe tener en cuenta que la implementación de esta modificación requerirá una carga extra de acceso a memoria. 
Por lo tanto, hay que analizar las alternativas posibles para que estos accesos se logren de manera eficiente.
Incluso teniendo esto en cuenta, la forma de implementar y utilizar esta tabla debería ser bastante directa, pero es necesario hacer un análisis exhaustivo de los posibles errores que se comenten al hacer los cálculos


\section{Calculo usando tablas de valores de la derivada}

Como se mencionó en el capítulo 1, el resultado más importante de la simulación es la trayectoria resultante ya que es la base de la simulación y se debe calcular siempre.
Dado que el valor mas importante que se obtiene a partir del potencial de L-J es la fuerza resultante de la interacción, se podría pensar que sería mas conveniente mantener en una tabla directamente los valores de la derivada. 
De esta forma, para obtener la fuerza correspondiente se busca el valor en la tabla asociado a la distancia \textit{r} entre las partículas (o el valor tabulado mas cercano). 

