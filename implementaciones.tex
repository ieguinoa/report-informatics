\chapter{Implementaciones}

\begin{comment}
Aca describo el kernel y aspectos relevantes de las versiones implementadas:
-que variables tengo que tener en cuenta en las implementaciones(por ej. el tamaño de la tabla,el rango, etc)
-que resultados espero!
-como afecta la periodicidad. la describo aca o en la parte de resultados??
-como afecta el tamaño de los bloques
\end{comment}

En este capítulo se describen los kernels implementados, incluyendo las variantes evaluadas. Además, se detalla cómo se tuvieron en cuenta los aspectos de performance expuestos en el capítulo 2.



\section{Consideraciones previas}

Una de las primeras decisiones que se deben considerar es la precisión que se utilizará para realizar los cálculos y acumular resultados. Dadas las caracterísiticas de la arquitectura GPU, esta decisión afectará la eficiencia de cualquier cálculo a realizar y por lo tanto es uno de los factores mas importantes a analizar.

Si bien es un factor importante, este ya ha sido estudiado usando otras implementaciones del método.  En particular, se ha estudiado usando Amber\cite{le2013spfp}. El trabajo mencionado analiza un esquema mixto que implica almacenar los calculos de fuerzas entre particulas utilizando precisión simple y acumular en variables de precisión doble la suma de todos los aportes que componen la fuerza sobre una partícula. Asumiendo los resultados en performance y correctitud encontrados en Amber y,
dado que es el software con el cual evaluaremos luego la calidad numérica lograda, utilizaremos, entonces, un esquema similar de precisión para nuestra implementación.



\section{Esquema general de la implementación sobre GPU}
El esquema general utilizado para implementar el algoritmo sobre GPU es el explicado en \cite{friedrichs2009accelerating}. Sin embargo, en el trabajo mencionado, el potencial incluye componentes de interacción de unión entre partículas, 
por lo que los calculos requeridos son algo distintos. En esta sección se explica, entonces, cuales son las variantes que utilizamos para realizar cada paso.

Para el primer paso de calculo de distancias entre partículas, la implementación es bastante directa. La granularidad usada es de 1 thread por cada par de partículas y se usaron tamaños de bloques fijos dd 1024x1.
Usando este esquema, con cualquier placa actual es posible lanzar en paralelo el calculo de todas las distancias para sistemas de hasta Xxx partículas. Teniendo en cuenta  que  el cálculo a realizar es muy simple, no presenta posibilidades de divergencia y dado el esquema de acceso a memoria, el kernel hace un uso óptimo de los recursos.

El paso correspondiente al calculo de las fuerzas entre particulas es quizas el mas complicado de implementar, y el numero de implementaciones existentes en la literatura es muy diversa, principalmente debido a que los calculos a realizar dependen de las interacciones que se estan utilizando y estás dan lugar a una gran variedad de posibles optimizaciones
\cite{friedrichs2009accelerating}\cite{gotz2012routine}\cite{salomon2013routine}
(citar los papers de amber y el de Accelerating Molecular Dynamic Simulation on Graphics Processing Units)

En el caso de interacciones non-bond, como las que se analizan en nuestro trabajo, la principal optimizacion deriva de la propiedad simetrica que presenta el potencial de interaccion, y por lo tanto la fuerza resultante, son simetricas

En nuestra implementación, esta parte es central ya que da lugar luego a diversas modificaciones. El presente trabajo, entonces, intenta seguir una implementacion estandar de forma tal que las modificaciones evaluadas tengan un carácter mas general y no dependan de ninguna

Para hacer uso de la propiedad de simetria, entonces, el calculo de la fuerza resultante de la interaccion solo se realiza sobre N2/2 pares, y el resto se deriva negando este valor.

Para descomponer las componentes el cálculo se realiza usando un thread por cada par de partículas.
Y para hacer la suma de la fuerza resultante, este se hace directamente sumando todos los valores de una fila. El acceso en este caso es coalescente ????? Y si hago una reducción ???

Los pasos siguientes son, bastante directos.
La aceleración se calcula en base a la fuerza resultante usando un thread por cada partícula.
La actualización de coordenadas también se realiza usando un thread para cada partícula. En este caso el único valor q se lee es el de



\section{Implementación usando tabla de valores potenciales}
\section{Implementación usando tabla de derivadas}
\section{Implementacion sobre CPU}
% \subsubsection{Tabla sobre memoria global}
% \subsubsection{Tabla sobre memoria de textura}
Con el fin de poder hacer una comparación real de los aspectos asociados a la performance se desarrolló además una implementación que realiza el cálculo de fuerzas(o solo el cálculo de derivadas???) sobre CPU.

Esta implementación solo modifica la forma en que se calcula la derivada del potencial, la cual se ejecuta ahora sobre la CPU, por lo tanto los datos de distancia y potenciales deben estar accesibles en memoria para que este cálculo pueda ser realizado.

Se implementaron variantes equivalentes para el calculo usando la ecuación derivada, usando una tabla de valores potenciales, y usando una tabla de valores de derivada.
Deberán tenerse en cuenta a la hora de usar esta implementación todos los aspectos asociados a la transferencia de datos entre la memoria del dispositivo y la memoria del host. En el próximo capitulo se verá como esta implementación es utilizada para demostrar las ventajas q aporta la arquitectura GPU.


% ****ACA PONGO LAS CARACTERISTICAS QUE OFRECE LA GPU Y QUE PUEDEN SER USADAS APROVECHANDO QUE HOY EN DIA LA MAYORIA DEL SOFT. ESTA IMPLEMENTADO SOBRE GPU*****
Para implementar este mecanismo hay varias propiedades del sistema de memoria que se puede usar y que se deben analizar......................................................
la memoria de textura provee el ajuste automatico en los limites de la tabla. Además, permite obtener el valor resultante de la interpolacion, lo cual si bien no es necesario, ayuda considerablemente en la precision.



\section{Implementación usando tabla de derivadas}
% \subsubsection{Tabla sobre memoria global}
% \subsubsection{Tabla sobre memoria de textura}


