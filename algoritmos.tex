
\chapter{Algoritmos}


En este capítulo, los conceptos del m\'etodo de dinámica molecular presentados en la introducción son explicados en términos del algoritmo que los representa,
haciendo énfasis en el paralelismo intrínseco que poseen, intentando abstraerse de cualquier implementación.

En la primera sección se describen loa conceptos del algoritmo base, describiendo todos los pasos de este, en particular todas las ecuaciones que no fueron detalladas en la introducción. 

En las siguientes secciones del capítulo se explican las modificaciones que este trabajo introduce y las propiedades que se deben tener en cuenta para implementarlas utilizando arquitecturas paralelas.

Es importante destacar que las adaptaciones que se plantean y analizan en este trabajo constituyen una modificación puntual para realizar el calculo del potencial de Lennard-Jones y, por lo tanto, 
tienen la capacidad de ser incluidas directamente en cualquier implementación existente y combinadas junto con cualquier otro método de optimización.
% Por otro lado, como se vió en el capitulo 1, existen pocas variantes para optimizar el cálculo de este potencial de no-unión y, dado que comprende el mayor aporte al costo computacional del método, 
% estas modificaciones pueden tener un gran impacto en la .


%Dado que el aporte a las interacciones que nos interesa es el correspondiente al potencial de L-J , este es el unico que se va a considerar. 
%Es decir, el campo de fuerzas estará definido exclusivamente por el potencial de Lennard-Jones que modela interacciones de no-unión. 
%De esta forma, el sistema evoluciona en el tiempo a partir de las velocidades iniciales de las particulas y las fuerzas que se derivan de este potencial, unicamente. 
% Sin embargo, usando el modelo de interacción resultante no será posible simular (de forma correcta) sistemas moleculares que incluyan enlaces químicos, ya que estos no estarán correctamente modelados por el campo de fuerzas. 
% El objetivo de esto es, en primer lugar, facilitar la implementacion de modificaciones/optimizaciones en el algoritmo relacionadas con el calculo del potencial de L-J.
% En segundo lugar, utilizando una implementacion propia se puede evaluar las ventajas y desventajas del uso de una tabla numerica de forma independiente a cualquier otra implementacion existente. 
% Se podrá verificar facilmente que las variaciones en el tiempo y en la precision de los resultados son producto exclusivo de las modificaciones implementadas en cada paso.




\section{Esquema general del algoritmo}

El algoritmo base sigue la especificación que se detallo en el capítulo 1(figura \ref{esquemaMD}). 

El primer paso de la iteración implica calcular la fuerza que actúa sobre cada partícula. 
El cálculo de la fuerza entre cada par de partículas responde a la siguiente ecuación: \begin{equation} \label{fuerza}\vec{F}(r)=\dfrac{\partial U(r)}{ \partial(r)}\end{equation}
                                                                                                                                                   
Por lo tanto, para calcular esto de forma discreta, primero debemos conocer el valor de r, es decir, la distancia entre cada par de partículas.

Este paso es claramente paralelizable ya que las posiciones son totalmente independientes entre si.
A partir de las distancias entre las partículas, la ecuación \ref{fuerza} permite obtener la fuerza resultante de la interacción entre un par de partículas, conociendo la función potencial.
Este valor corresponde a la fuerza en la dirección de $\vec{r}$ , es decir, en la dirección del vector distancia. 
Para poder obtener la suma total de las fuerzas que actúan sobre una partícula, es decir, como resultado de interacciones con todas las partículas del sistema,
es necesario sumar todos los aportes. Esto implica sumar fuerzas con distintas direcciones, lo que se traduce en una suma de los componentes x,y,z de los distintos vectores. 
Por lo tanto, el cálculo de la fuerza total que actúa sobre una partícula se divide en tres pasos:

\begin{itemize}
\item En un primer paso se calcula el valor de la fuerza entre cada par de partículas de acuerdo a la ecuación \ref{fuerza}.

\begin{equation} 
  \dfrac{\partial(U(r)) }{ \partial(r)}  =  24\epsilon \bigg(\dfrac{{\sigma}^{6}}{{r}^{7}} - \dfrac{2{\sigma}^{12}} {{r}^{13}}\bigg)
\end{equation}


\item En un segundo paso se descompone esta fuerza en sus 3 componentes (x,y,z): \begin{equation}                                                                                
F_i=\vec{F}(r)\dfrac{\partial r}{\partial i}  \hspace{30pt} \text{para i= x,y,z}
                                                                                 \end{equation}

\item Por último se realiza la suma de los componentes para todas las fuerzas que actúan sobre una partícula: 
\begin{equation}
F_i=\sum F_i
\end{equation}

\end{itemize}



% CALCULO DE LA ACELERACION/CAMBIO EN LA VELOCIDAD
Una vez obtenida la fuerza que actúa sobre cada partícula(componetes x,y,z), se puede calcular la aceleración resultante de ésta y actualizar la velocidad de la partícula de acuerdo a este cambio.
La velocidad estará dada por:
 \begin{equation} 
    v(t+dt)= v(t) + (\frac{F*dt}{masa})
% velocity[i] = Vt + ( (force[i]*dtx) / m[type] );
 \end{equation}

Nuevamente, la velocidad depende de propiedades de la partícula (velocidad anterior, fuerza y masa) y el cálculo es totalmente independiente para las distintas partículas.  


% PASO DE ACTUALIZACION DE CORDENADA
El proceso de actualización de coordenadas deberá calcular para cada partícula la nueva posición, la cual estará dada en funcion de la posicion y velocidad previa. La ecuación que se debe calcular en este paso es:
\begin{equation} 
    Pos_x(t+dt)= Pos_x(t) + (v_x*dt)
% positions[i] = positions[i] + (velocity[i] * dtx);
\end{equation}

Claramente este cálculo es totalmente independiente entre las partículas y por lo tanto puede hacerse completamente en paralelo.





% ACA DEBERIA HACER UN BALANCE GLOBAL DE LOS PUNTOS CRITICOS DEL ALGORITMO, O ALGO COMO PARA INTRODUCIR LAS MODIFICACIONES PLANTEADASme
%Esto pasarlo a la parte donde explico el paso de calculo de la derivada. También debería aclarar que el paso de cálculo de fuerzas y el del potencial (si es requerido) tienen una complejidad similar 
Las limitaciones en este conjunto de pasos se deben a que tiene un orden de ejecucion $N^2$. Aún en arquitecturas altamente paralelas como las GPUs, ,este cálculo es extremadamente costoso. 
Aun cuando el numero de procesadores en estas arquitecturas siga creciendo, es poco probable que se supere el numero de calculos que es necesario hacer para simular sistemas grandes.
Es necesario, entonces, realizar aproximaciones extra para superar este limite.




\section{Calculo usando tablas de valores del potencial }

El algoritmo utiliza este potencial para obtener la interaccion entre cada par de particulas y derivar la fuerza resultante de esto. 
Ademas, esta funcion permite evaluar la energia potencial total asociada a la conformacion. Para esto se realiza la suma del valor de la funcion entre todos los pares de particulas.

Se debe tener en cuenta que la implementación de esta modificación requerirá una carga extra de acceso a memoria. 
Por lo tanto, a la hora de realizar la implementación hay que analizar las alternativas posibles para que estos accesos se logren de manera eficiente.
 


% 

% LUEGO DECIR QUE DADAS LAS CARACTERISTICAS DE LA FUNCION DE L-J SERIA RAZONABLE PENSAR EN USAR UNA TABLA PARA CALCULAR EL POTENCIAL DE FORMA AISLADA(CALCULAR EL POTENCIAL UNA VEZ)
% QUE PASARIA SI APROVECHAMOS ENTONCES ESTO PARA IMPLEMENTAR UN SISTEMA DE POTENCIALES TABULADOS DENTRO DE AL ETAPA DE CALCULO EN EL ALGORITMO QUE SIGUE EL ESQUEMA GENERAL DE DM SOBRE GPU,
% CON EL FIN DE REDUCIR EL COSTO DEL CALCULO y OPTIMIZAR LA SIMULACION

%esto pasarlo a la sección q habla de la tabla directamente
Basandose en la forma funcional del potencial, es razonable pensar que, utilizando una tabla conteniendo los resultados para un rango de valores definido y, aun sin tener ésta un tamaño excesivo, se podria obtener una buena aproximacion del resultado para cualquier distancia. 
 

La primera aproximacion que involucra el uso de tablas se deduce de forma bastante directa a partir de la descripción general del algoritmo, se basa en utilizar una tabla que tenga almacenados valores del potencial para un rango definido de distancias.
Teniendo este tipo de valores almacenados es posible utilizarlos para calcular tanto las fuerzas resultantes (necesarias para la evolucion de la simulacion) como el valor de la enegia potencial total, el cual puede ser requerido por el usuario como resultado de la simulacion.


 
%  ACA PUEDO DECIR QUE LA TABLA PUEDE SER IMPLEMENTADA EN CUALQUIER SISTEMA DE MEMORIA. ES DEICR, SI BIEN SE PUEDE IMPLEMENTAR SOBRE CPU ESTA SOLUCION (Y DE HECHO LO HICE) USAMOS LA GPU POR LA MEMORIA DE TEXTURA Y POR SER HOY EN DIA UN ESTANDAR EN LO QUE RESPECTA A DINAMICA MOLECULAR
si bien el objetivo de este trabajo es optimizar la ejecucion de las implementaciones sobre GPU, la solucion aqui planteada puede ser implementada en distintas arquitecturas.

% Aca explico masomenos como seria la tabla: Tengo que explicar que el potencial de Lennard Jones depende del par de elementos que interaccionan (y por lo tanto va a haber tablas de LxL )
% DECIR MASOMENOS CUANTOS TIPOS DE ELEMENTOS EXISTEN , ETC
Como se vió en el capítulo 1, la ecuación del potencial de Lennard-Jones (ecuación \ref{lennardEquation}) depende directamente de \textit{r}, y los valores de $\epsilon$ y $\sigma$ de ambas partículas.
La tabla que se utilizará tendrá, entonces, tres dimensiones. En una dimensión el valor del potencial variará en función de \textit{r}.
Las otras dos dimensiones representan una matriz que contiene todas las combinaciones de pares $\epsilon$-$\sigma$.

Utilizando esta tabla, para obtener el valor de la energia potencial de interaccion entre un par de particulas simplemente se recupera el valor asociado a la distancia(r) que existe entre las particulas(o el valor tabulado mas cercano).
Para poder obtener la fuerza resultante de esta interaccion es necesario conocer el valor de la derivada de la funcion potencial para la distancia correspondiente.
Utilizando la tabla descita previamente se puede obtener el valor de la derivada mediante la tecnica de derivacion numerica. 
Para esto se obtienen primero dos valores de potencial a partir de la tabla, correspondientes a distancias de r+-x. Donde x es un valor discreto definido en base a.....
Como se ve, esto implica una doble fuente de error, debido a una aproximacion tanto en la utilizacion de la derivada numerica como en la obtencion de los valores de potencial a partir de una tabla.


% Esta linea no se si va aca o la pongo directamente en la parte de implementacion
El esquema general del algoritmo se mantiene en todas las variantes. Las modificaciones estan centradas en la etapa de calculo del potencial, el contexto de este se mantiene siempre igual.



\section{Calculo usando tablas de valores de la derivada  }

Como se mencionó en el capitulo 1, el resultado mas importante de la simulación es la trayectoria. Además, es la base de la simulacion y se debe calcular siempre.
Dado que el valor mas importante que se obtiene a partir del potencial de L-J es la fuerza resultante de la interacción, se podría pensar que sería mas conveniente mantener en una tabla directamente los valores de la derivada. 
De esta forma, para obtener la fuerza correspondiente se busca el valor en la tabla que se encuentre mas cercano a la distancia de interacción. 



%ESTO NO SE SI PONERLO ACA O DIRECTAMENTE CUANDO ESTOY INTRODUCIENDO LA PARTE DE RESULTADOS
Se debe tener en cuenta que, aún cuando sea posible realizar una buena aproximacion utilizando tablas, el algoritmo de dinamica molecular implica utilizar este valor para obtener las fuerzas y movimientos resultantes de la interaccion, lo cual afecta la posicion (y por lo tanto el valor del potencial asociado) en la proxima iteracion. 
De esta forma, los errores resultantes de la aproximacion por utilizar tablas de valores seran propagados a lo largo de una simulacion que involucra miles de pasos.....


